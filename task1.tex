
\documentclass[12pt]{article}

\usepackage[russian]{babel}

\title{Домашняя работа №1}
\author{Совин Кирилл}
\date{}

\begin{document}
	\maketitle
\begin{flushright}
\textit{Audi multa,\\
	loquere pauca}	
\end{flushright}

\vspace{20pt}

Это мой первый документ в компьютерной вёрстке \LaTeX.

\begin{center}
\huge{ \textsf{<<Ура!!!>>} }\\
\end{center}

\par А теперь формула. \textsc{Формула} --- краткое и точное словесное выражение, определение или же ряд математических величин, выраженный условными знаками.
\vspace{15pt}

\hspace{28pt}  {\Large\textbf{Термодинамика}}

\par Уравнение Менделеева--Клапейрона --- уравнение состоя
ния идеального газа, имеющее вид $pV = \nu pV$, где $p$ --- давление, $V$ --- объём, занимаемый газом,$T$ --- температура газа, $\nu$ --- количество вещества газа, а $R$ --- универсальная газовая постоянная.
\vspace{15pt}

\hspace{28pt}  {\Large\textbf{Геометрия \hfill Планиметрия}}

\par Для \textit{плоского} треугольника со сторонами $a, b, c$ и с углом $\alpha$, лежащим против стороны $a$, справедливо соотношение
$$
a^2 = b^2 + c^2 - 2bc \cos \alpha,
$$
из которого можно выразить косинус угла треугольника:
$$
\cos \alpha =\frac {b^2 + c^2 - a^2} {2bc}.
$$

Пусть $p$ --- полупериметр треугольника, тогда путём неслождных преобразований можно получить, что

$$
\tan \frac \alpha 2 = \sqrt{\frac {(p-b)(p -c)} {p(p-a)} }.
$$

\begin{flushleft} % а есть способ прижать строку к началу поизящнее?
На сегодня, пожалуй, хватит\dots Удачи!
\end{flushleft}

\end{document}
